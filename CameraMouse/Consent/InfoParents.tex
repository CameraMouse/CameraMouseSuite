% Form for parents/guardians of minors

\documentstyle[10pt,psfig]{letter}
\setlength{\textheight}{9.5in}
\setlength{\footheight}{0.0in}
\setlength{\topmargin}{0in}   % should be 0
\setlength{\headheight}{0.0in}
\setlength{\headsep}{0.0in}
\setlength{\textwidth}{6.6in}
\setlength{\oddsidemargin}{0in}
\setlength{\parindent}{0pc}
\setlength{\parskip}{6pt plus 1.5pt minus 1.5pt}

\name{Margrit Betke}

\signature{\vspace*{-.7cm} Margrit Betke, Ph.D. \\ Associate
Professor}

\begin{document}
\pagestyle{empty}

\begin{letter}

%\bucsletterhead

%\vspace*{2cm}

{\large \centerline{\bf Information about Research Study}

%\vspace*{.5cm}

Dear Parent,}

I would like to ask you to consider allowing your child to participate
in a research study entitled ``Video-Based Computer Interfaces for
People with Severe Disabilities.''  Your child's participation would
help me to improve a computer interface for people with motion
impairments.  Such an interface may enable your child or other
children to use education software.

I would like to introduce myself to you.  My name is Dr.\ Margrit
Betke.  I am an Associate Professor of Computer Science at Boston
University.  I have been working with the Campus School at Boston
College since 1999 and have co-developed the ``Camera Mouse'' with
Prof.\ James Gips at Boston College.  Your child may already be
familiar with the Camera Mouse.  It is an interface system that tracks
the computer user's movements with a video camera and translates them
into the movements of the mouse pointer on the screen.  Computer users
with severe motion impairments may be able to use this
mouse-replacement technology to interact with educational software.

The goal of my current research study is to improve the Camera Mouse
so that the mouse pointer can be tracked reliably and
automatically. Such an improvement might make it easier for your child
or others to interface with the computer.  To determine the situations
in which the current Camera Mouse does not work optimally for your
child, I would like to study video recordings of your child using the
Camera Mouse.  I am therefore requesting your consent about two
matters:

\begin{enumerate}
\item Your consent for your child to participate in a computer
session, in which he or she uses the Camera Mouse with assistive
software.
\item Your consent for me to take video recordings of your child
during this computer session.
\end{enumerate}

To obtain your consent about these two matters, I am attaching two
forms to this letter for you to sign.  Please note your consent is
entirely voluntary and can be withdrawn at any time.  If you refuse to
participate in the research study, your child's status as a student at
the Campus School will not be affected.

The attached video release form offers you a choice for your level of
consent.  Please sign the first statement on the form if you wish for
your child to participate in the research study but do not wish
anybody except me and my research assistants to view the video data.
Please sign the second statement, if you wish for your child to
participate in the research study and if you also allow us to include
his or her picture or video in our research and education publications
and demonstrations.  Such publications may help other computer
scientists to better understand how to develop assistive technology
for children with motion impairments.

The risks of this research for your child are low because the
professional caregivers who already serve your child at the Campus
School will always be present during the computer session.  They will
monitor your child carefully and let me know when I should stop the
computer session.

If you have further questions, please call me at 617-353-6412 or
617-353-8919 or email me at betke@cs.bu.edu.
\begin{center}
\closing{Sincerely,
}
\end{center}

\end{letter}

\end{document} 