% Form for parents/guardians of minors

\documentstyle[10pt,psfig]{letter}
\setlength{\textheight}{9.75in}
\setlength{\footheight}{0.0in}
\setlength{\topmargin}{-0.25in}   % should be 0
\setlength{\headheight}{0.0in}
\setlength{\headsep}{0.0in}
\setlength{\textwidth}{6.6in}
\setlength{\oddsidemargin}{0in}
\setlength{\parindent}{0pc}
\setlength{\parskip}{6pt plus 1.5pt minus 1.5pt}

\name{Margrit Betke}

\signature{\vspace*{-.7cm} Margrit Betke,\\ Associate Professor}

\begin{document}
\pagestyle{empty}

\begin{letter}

%\bucsletterhead 
\centerline{\Large Informed Consent for a Minor to
Participate in a Research Study }

\vspace*{.5cm}
{\em Title of Research Study:} Video-Based Computer Interfaces for People
with Severe Disabilities

{\em Investigator:} Margrit Betke, Ph.D.\\
Phone: 617-353-6412, Email: betke@cs.bu.edu

{\em Purpose:} The objective of this project is to help people with
severe disabilities gain access to a computer and thereby obtain a
tool for communication.  The project will explore methods for
detection, tracking, and interpretation of human body components and
their motion in video.  We will work with a computer program, called
Camera Mouse, which tracks the computer user's movements with a video
camera and translates them into the movements of the mouse pointer on
the screen.  Computer users with severe motion impairments may be able
to use this mouse-replacement technology to interact with educational
software.

If you agree to allow your child/ward to participate in this research
project, the following things will happen:

\begin{itemize}
\item Your child/ward will be videotaped at \rule{3cm}{.01cm} in at
 most three computer sessions, which will take between 10 to 30
 minutes.  Typically there will be one week between each session.

\item During the sessions, your child/ward will be asked to control
the mouse pointer on the screen with the Camera Mouse by small
movements of the head, toe, thumb, or other body parts.

\item If your child/ward is able to control the mouse pointer, we will
ask him or her to perform a sequence of communication tasks.  The
tasks may be writing a sentence with custom-built spelling software or
playing with educational entertainment software.

\item A caregiver will have to be present throughout the testing and
observe the child/ward to verify that he or she assents to
participate.

\end{itemize}

{\em Confidentiality:} All information and video data obtained in this
research project will be considered confidential.  It will be used for
research purposes only.  All records will be coded by an id number and
kept in a locked file at Boston University.  Access to the digital
data will be restricted.

{\em Benefits:} This study introduces participants to a new computer
tool for communication and education.

{\em Risks:} The computer session may be tiring for your child.  The
risk of exertion will be minimized by limiting the computer sessions
to at most 30 minutes. We may use lighting for the videotaping, and sometimes this
lighting may be warm.

{\em Questions:} The researcher has provided information to you about
this study and has offered to answer your questions.  If you have
further questions, contact Prof.\ Margrit Betke at 617-353-6412.  If
you have any questions about your rights as a participant in a
research study, contact Dr.\ David Berndt, Administrator of the
Institutional Review Board, Boston University, at 617-353-4365.

{\em Right to Refuse or Withdraw:} Your and your child/ward's
participation in the research study is entirely voluntary.  You are
free to refuse to take part or withdraw at any time.  If you refuse to
participate, you and your child/ward will not lose any benefits to
which your are otherwise entitled.

{\em Consent:} I agree to allow my child/ward, \hrulefill, to
participate in this research study.

Parent/Guardian's Signature \hrulefill  Date: \hrulefill

Signature of Investigator Obtaining Consent \hrulefill  Date: \hrulefill

\end{letter}

\end{document} 